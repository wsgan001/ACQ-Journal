\subsection{CS on profiled graphs}
\label{PCQ}

{\bf ACQ limitations.}
One of ACQ's key properties is keyword cohesiveness which may bring some limitations. 

$\bullet$ First of all, keyword cohesiveness requires that attributed communities generated should have common keywords according to the query keyword set pre-defined by users. Any inaccurate description and spelling error in the query keywords of ACQ may cause the interference and even failure of searching targeted communities. 

$\bullet$ Furthermore, keyword cohesiveness constrains that vertices in communities should share the common keywords regardless of semantically related keywords. That may result in ignoring some users even if they have similar keywords. In practical, keywords are tightly or loosely related which means relationship and semantic meaning of keywords should be considered, however, ACQ can not satisfy the requirement.

Thus, we propose a tree structure called profile tree to describe the semantic relationship of communities' attributies instead of the flattened keyword attributes.

{\bf Profile tree.} 
A profile tree (P-tree) $T=\{V_T,E_T\}$ is a set of attributes. $V_T$ is the set of tree nodes. One distinguished node $r\in V_T$ denotes the root of $T$. $l:V_T\to A$ is a function that maps tree nodes one-to-one to a set of attributes $A=\{a_1, a_2, a_3, ......, a_n\}$, i.e., $\forall x\in V_T$, $l(x)=a_x$. $E_T=\{(x,y)|x,y\in V_T\}$ is the set of edges that represent relationship between attributes, such that $\forall (x,y)\in E_T$, $l(y)$ is the semantic sub-attributes of $l(x)$. Each edge of the P-tree is unique, i.e., $\forall y \in V_T, y\neq r, \exists !x\in V_T, s.t.(x,y) \in E_T$.

{\bf Induced rooted subtree.}
After introducing P-tree, we give the definition of Induced rooted subtree.
Give two P-trees $T=(V_{T},E_{T})$ rooted at $r_T$ and $S=(V_S,E_S)$ rooted at $r_S$. $S$ is the induced rooted subtree of $T$ when there exist a one-to-one map $\varphi: V_S \to V_T$, such that:

$\bullet$ $\varphi(r_S)=r_T$;

$\bullet$ $ \forall x \in V_S, l(x)=l(\varphi(x))$;

$\bullet$ $(x,y) \in E_S$, iff $(\varphi(x),\varphi(y)) \in E_T$;

{\bf Maximal induced rooted subtree.}
Given a database of P-trees $D$, P-tree $S$ is the maximal induced rooted subtree of $D$ if and only if when there does not exist a P-tree $T \in D$ such that $S$ is the induced rooted subtree of $T$. 

We now formally define the PCQ problem as follows. 

\begin{problem}[PCQ]
Given a profiled graph $G$, a positive integer $k$, and one query node $q \in G$, return a set $\mathcal {G}$ of graphs, such that $\forall G_q \in \mathcal {G}$, the following properties hold:

\vspace{1ex}
$\bullet$ \textbf{Connectivity.} $G_q$is connected and $G_q \subseteq G$;

$\bullet$ \textbf{Structure cohesiveness.} $\forall v\in G_q$, $deg_{G_q}(v)\geq k$;

$\bullet$ \textbf{Semantic cohesiveness.} all the vertices in $G_q$ share the maximal common induced rooted subtree of $q$;
\end{problem}

According to the ACM Computing Classification System (CCS) 2012 version, academic terms concerning Computer Science can be constructed in a hierarchical structure. We use it to illustrate the PCQ problem in Example~\ref{pcqEg}.  

\begin{example}
\label{pcqEg}
Consider the graph in Figure~\ref{fig:eg1}. Vertices represent researchers and edges represent their co-author relationship. Their individual expertises are used to represent the attributes. The complete P-tree is shown in Figure~\ref{fig:profile}. Let $k=2$, $q=A$. Then the targeted community of PCQ is $\{A,D,E,F\}$, and their common maximal induced tree is \{root, IS, CM\}. 
\end{example}

{\bf Technical challenges.}
PCQ Problem mainly has following technical challenges.

$\bullet$ Mining maximal common trees in tree database is not easy to implement. In special case, suppose the tree is a binary tree, number of its subtrees with $n$ tree nodes has $\frac{C(2n,n)}{n+1}$ which asymptotically grows as $\frac{4^n}{n^{\frac{3}{2}\sqrt \pi} }$. 

$\bullet$ Maximal frequent subtree pattern mining is np-hard(cite The complexity of mining maximal frequent itemsets and maximal frequent patterns).

Thus, one baseline approach is we first search $k-\widehat{core}$ containing $q$. Then we iteratively mining common profile trees incrementally or decrementally until the maximal P-trees are found. Finally, the targeted communities are returned. 

\begin{figure}[ht]
\label{pcqFigure}
    \centering
    \mbox{
        \subfigure[graph]{
            \includegraphics[width=.55\columnwidth]{figures/eg1}
            \label{fig:eg1}
        }
        \hspace{0ex}
        \subfigure[p-tree]{
            \includegraphics[width=.4\columnwidth]{figures/profile}
            \label{fig:profile}
        }
    }
    \caption{A profiled graph.}
\end{figure}

