\section{Proofs of Lemmas}
\label{app:proof}

% go back to the original orders
\addtocounter{lemma}{-1}
\addtocounter{lemma}{-1}
\addtocounter{lemma}{-1}
\addtocounter{lemma}{-1}
\addtocounter{equation}{-1}
\addtocounter{equation}{-1}

\begin{lemma}[Anti-monotonicity]
  \label{lemma:apriori-app}
  Given a graph $G$, a vertex $q\in G$ and a set $S$ of keywords, if there exists a subgraph $G_k[S]$,
  then there exists a subgraph $G_k[S']$ for any subset $S'\subseteq S$.
\end{lemma}

\begin{proof}
Based on the definition of $G_k[S]$, each vertex of $G_k[S]$ contains $S$.
Consider a new keyword set $S'\subseteq S$.
We can easily conclude that,
each vertex of $G_k[S]$ contains $S'$ as well.
Also, note that $q\in G_k[S]$.
These two properties imply that there exists one subgraph of $G$,
namely $G_k[S]$, with core number at least $k$,
such that it contains $q$ and every vertex of it contains keyword set $S'$.
It follows that there exists such a subgraph with maximal size (\textit{i.e.}, $G_k[S']$).
\end{proof}

\begin{proposition}\label{prop:pre-app}
  For any keyword set $S$, and vertex $q$,
  if $G_k[S]$ exists, then $G_k[S]\subseteq G_k[S']$ for any subset $S'\subseteq S$.
\end{proposition}

\begin{proof}
Since $G_k[S]$ contains vertex $q$ and every vertex in $G_k[S]$ contains $S'$ (due to $S'\subseteq S$), then $G_k[S]\cup G_k[S']$ also contains vertex $q$ and every vertex in it contains $S'$. In addition, the core numbers of $G_k[S]$ and $G_k[S']$ are at least $k$, it follows that the core number of $G_k[S]\cup G_k[S']$ is at least $k$.
Based on the definition of $G_k[S']$, we have $G_k[S]\cup G_k[S']\subseteq G_k[S']$. It follows that $G_k[S]\subseteq G_k[S']$.
\end{proof}


\begin{lemma}
\label{lemma:coreDown-app}
  Given two subgraphs $G_k[S_1]$ and $G_k[S_2]$ of a graph $G$,
  for a new keyword set $S'$ generated from $S_1$ and $S_2$ (\textit{i.e.}, $S'=S_1\cup S_2$),
  if $G_k[S']$ exists, then it must appear in a $k$-$\widehat {core}$ with core number at least
  \begin{equation}
    max\{core_G[G_k[S_1]], core_G[G_k[S_2]]\}.
  \end{equation}
\end{lemma}

\begin{proof}
Since $S'$ is generated from $S_1$ and $S_2$, then $S_1\subseteq S'$ and $S_2 \subseteq S'$. Based on Proposition~\ref{prop:pre-app}, we have $G_k[S']\subseteq G_k[S_1]$. With such a containment relationship, it follows that $min\{core_G[v]|$ $v\in G_k[S_1]\}\leq min\{core_G[v]|v\in G_k[S']\}$. Hence, the core number of $G_k[S']$ is at least the core number of $G_k[S_1]$. Formally, $core_G[G_k[S_1]]$ $\leq core_G[G_k[S']]$. For the same reason, $core_G[G_k[S_2]]\leq core_G[G_k[S']]$. It directly follows the lemma.
\end{proof}



\begin{lemma}
\label{lemma:coreExist-app}
  Given a connected graph $G(V,E)$ with $n$=$|V|$ and $m$=$|E|$,
  if $m - n < \frac{{{k^2} - k}}{2} - 1$, there is no $k$-$\widehat {core}$ in $G$.
\end{lemma}

\begin{proof}
From Definition~\ref{def:kcore}, we can easily conclude that,
for any specific $k$, a $k$-$\widehat {core}$ has at least $k$+1 vertices.
Since each vertex in a specific $k$-$\widehat {core}$ has at least $k$ edges,
the minimum number of edges in a $k$-$\widehat {core}$ is $\frac{{(k + 1)k}}{2}$.

Consider a connected graph, which contains a $k$-$\widehat {core}$ and has the minimum number of edges,
where the $k$-core contains only $k+1$ vertices and
all the rest $n-(k+1)$ vertices are connected with this $k$-$\widehat {core}$.
The total number of edges is
\begin{equation}
\frac{{(k + 1)k}}{2} + \left[ {n - (k + 1)} \right] = m
\end{equation}

By simple transformation, we can conclude that,
if $m - n < \frac{{{k^2} - k}}{2} - 1$, there is no $k$-$\widehat {core}$ in $G$.
\end{proof}



\begin{lemma}
\label{lemma:kcoreIntersect-app}
  Given two keyword sets $S_1$ and $S_2$, if $G_k[S_1]$ and $G_k[S_2]$ exist, we have
  \begin{equation}
    G_k[S_1\cup S_2] \subseteq G_k[S_1]\cap G_k[S_2].
  \end{equation}
\end{lemma}

\begin{proof}
Based on Proposition~\ref{prop:pre-app} and $S_1\subseteq {S_1} \cup {S_2}$, we have ${G_k}[{S_1} \cup {S_2}]\subseteq {G_k}[{S_1}]$. For the same reason we have ${G_k}[{S_1} \cup {S_2}]\subseteq {G_k}[{S_2}]$. It directly follows the lemma.
\end{proof}
