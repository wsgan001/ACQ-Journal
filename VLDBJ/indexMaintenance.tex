{\color{blue}
\section{Index Maintenance}
\label{indexMaintenance}

%In practice, the keywords and edges of graphs are often frequently updated.
In practice, the graphs are continuously evolving. Thus keywords and edges of graphs are often frequently updated.
Clearly, when the graph is updated, both the CL-tree index and the ACQ query results also need to be updated.
A straightforward method for handling the dynamic graph is to rebuild the index from scratch when an update is made.
However, this method is very inefficient, especially when the updates are very frequent.
To alleviate this issue, in this section we study how to dynamically maintain the CL-tree index efficiently,
and propose algorithms for maintaining the CL-tree without rebuilding the CL-tree from scratch.

The update for keyword update, i.e., inserting or deleting a keyword from a vertex's keyword set, is easy to be handled, since we can simply find the CL-tree node containing the vertex and update its $invertedList$.
For the update of edge, i.e, inserting or deleting an edge, it is not straightforward how to accordingly update the CL-tree efficiently. This is because, the insertion or deletion of a single edge may trigger updates in several CL-tree nodes as well as their structures.
We first present how to handle keyword update in Section~\ref{sec:keyword}.
Then, we discuss the maintenance of CL-tree for the insertion and deletion of an edge in Sections~\ref{sec:edgeInsertion} and~\ref{sec:edgeDeletion}.

\subsection{Keyword Update}
\label{sec:keyword}
Recall that in the {\tt advanced} method (Section~\ref{advancedIndex}), we have built a vertex-node map, where each vertex is mapped to a CL-tree node. Note that we can build such a map by traversing the tree if we use {\tt basic}.
To insert a new keyword for a vertex $v$, we can first locate the CL-tree node, $p$, containing $v$ by the vertex-node map, and then insert the keyword and vertex ID into $p.invertedList$. To remove a keyword of a vertex, we can have a similar process on the CL-tree.

\subsection{Edge Insertion}
\label{sec:edgeInsertion}

As aforementioned, inserting an edge may trigger the updates of several CL-tree nodes as well as their structures.
We illustrate this by Example~\ref{eg:edgeInsertion}.

\begin{example}
\label{eg:edgeInsertion}
Consider the graph in Figure~\ref{fig:advancedIndex}. If we insert an edge ($H$, $G$) as shown in Figure~\ref{fig:coreNumber}, the core number of vertex $H$ increases to 2 and we need to move it down to a node in the lower level. If we insert an edge ($G$, $I$), the connectivity of some vertices changes as shown in Figure~\ref{fig:ConnectivityExmp} and thus the corresponding subtrees are merged as a new one.
\end{example}

\begin{figure}[ht]
    \centering
    \mbox{
        \subfigure[core number]{
            \includegraphics[width=.4\columnwidth]{figures/movedownExmp}
            \label{fig:coreNumber}
        }
        \hspace{2ex}
        \subfigure[connectivity]{
            \includegraphics[width=.4\columnwidth]{figures/connectiveExmp}
            \label{fig:ConnectivityExmp}
        }
    }
    \caption{The core number and connectivity change.}
    \label{fig:connectivity}
\end{figure}

To maintain the CL-tree for inserting an edge, we propose an algorithm called {\tt insertEdge}.
The main idea is that, we first find vertices whose core numbers change, then change their positions in the CL-tree, and merge some subtrees. Let $V^+$ be the set of vertices whose core numbers increase after inserting an edge ($u,v$).
We summarize the main steps of {\tt insertEdge} as follows.

$\bullet$ \textbf{Step 1:} Compute $V^+$;

$\bullet$ \textbf{Step 2:} Move down vertices of $V^+$;

$\bullet$ \textbf{Step 3:} Merge subtrees.

We now elaborate these steps one by one.

\textbf{Step 1: Compute $V^+$.}
Inserting an edge only affects the core numbers of a small number of vertices~\cite{kcoreUpdate,sariyuce2016incremental}. We first give a definition, a theorem and a lemma proposed in this paper.

\begin{definition}[\cite{kcoreUpdate}]
\label{df:inducedgraph}
Given a graph $G$ and a vertex $v$, the induced core subgraph of $v$, denoted as $G_v$, is a connected subgraph containing $v$ and the core numbers of all vertices in $G_v$ equal to $core_{G}[v]$.
\end{definition}

Notice that, the sets of vertices in $G_u$ ($G_v$) are actually subsets of vertices in $p_u.vertexSet$ ($p_v.vertexSet$),
where $p_u$, $p_v$ denote the nodes that contain $u$, $v$.

\begin{theorem}[k-core update theorem\cite{kcoreUpdate}]
\label{thrm:kcoreupdate}
Given a graph $G$ and two vertices $u$ and $v$. After inserting or deleting an edge $(u$,$v)$ in $G$, we have that,

$\bullet$ If $core_G[u] > core_G[v]$, only the core numbers of vertices in $G_v$ may need to be updated.

$\bullet$ If $core_G[u] < core_G[v]$, only the core numbers of vertices in $G_u$ may need to be updated.

$\bullet$ If $core_G[u] = core_G[v]$, only the core numbers of vertices in the union of $G_u$ and $G_v$, \textit{i.e.,} $G_{u\cup v}$ may need to be updated.
\end{theorem}

\begin{lemma}[\cite{kcoreUpdate}]
\label{lm:kcorelemma}
After inserting (deleting) an edge, the core number of any vertex in $G$ increases (decreases) by at most 1.
\end{lemma}

By above theorem and lemma, we can conclude that only a small number of vertices need to change their core numbers.
In specific, we can first find node $p_u$ ($p_v$) and then compute the vertex set $V^+$ in which vertices's core numbers increase by 1 using the algorithm in~\cite{kcoreUpdate}.

\textbf{Step 2: Move down vertices of $V^+$.}
Let $p$ be the node containing $V^+$ and $c$=min\{$core_G[u], core_G[v]$\}). Since the core numbers of vertices in $V^+$ increase by 1 (from $c$ to $c$+1), we need move them down to nodes in the lower level.
During the moving down process, we may also need to reorganize $p$'s child nodes. Let us illustrate this by Example~\ref{eg:goDown}.

\begin{figure}[ht]
    \centering
    \includegraphics[width=1.04\linewidth]{figures/movedownEmp}
    \caption{An example of the tree index update.}
    \label{fig:movedownEmp}
\end{figure}


\begin{example}
\label{eg:goDown}
Consider a graph in Figure~\ref{fig:movedownEmp}(a) and its CL-tree in Figure~\ref{fig:movedownEmp}(b).
Let us insert an new edge (8, 11). We first get $V^+$=\{8, 11, 23\} and $c$=2.
Next, we move them down from $r_1$ to $r_3$.
Besides, we have to merge $r_2$ into $r_3$ and place $r_4$ as $r_3$'s child node,
since their connectivity changes after the insertion.
The updated CL-tree is depicted in Figure~\ref{fig:movedownEmp}(c).
%Consider that an edge ($8,11$) is inserted in Figure~\ref{fig:movedownEmp}(a). Before insertion, the CL-tree index is shown in Figure~\ref{fig:movedownEmp}(b). After insertion, vertices $8$, $11$ and $23$ form $V^+$ and need to increase the core number from 2 to 3. Thus they need to move down a level and the tree index changes accordingly. Nodes $r_2$, $r_3$ whose core number are 3 are traced by the neighbors of the vertices ($11$ and $23$) and therefore $r_2$, $r_3$ are merged as a new node (see the arrows in Figure~\ref{fig:movedownEmp}(b)). Another node $r_4$ which is connected through $8$ is also affected and re-linked as a child node of the new node $r_2$ because the core number of $r_4$ is 4. Other nodes remain unchanged. Finally the updated tree index is presented in Figure~\ref{fig:movedownEmp}(c).
\end{example}

%We divide $p$'s child nodes into two sets $Z_1$ and $Z_2$, where nodes in $Z_1$ have a core number of $c$+1 and nodes in $Z_2$ have a core number of $c$+2 or more.

Clearly, moving down vertices of $V^+$ from $p$ to $p$'s child node (denoted by $p'$) may change the connectivity of $p$'s child nodes.
Consider a specific vertex $a$$\in$$V^+$ and we initialize two empty sets $B_1$ and $B_2$.
For each of $a$'s neighbor $b$ whose $core_G[b]$$\textgreater$$c$, we first find the node $p_b$ containing $b$,
and then trace it up from $p_b$ along the CL-tree until a child node of $p$, denoted by $o_b$.
If $o_b$ has a core number of $c$+1, we put it into $B_1$;
Otherwise, we put it into $B_2$.
Then, after moving down vertices of $V^+$,
nodes in $B_1$ should be merged into $p'$ and nodes in $B_2$ will be child nodes of $p'$.


%Some child nodes of $p$ which are connected through vertices in $V^+$ may also need to be updated. These child nodes can be divided into two types. (1) The core number of the node is $c$+1; (2) The core number of the node is larger than $c$+1. If $c=core_G[v]$, for example, after increasing the core number of the vertex $v$ to $c$+1, all ($c$+1)-$\widehat{core}$ which are connected with $v$ should be merged as a new ($c$+1)-$\widehat{core}$. Reflected in the tree index, the first type of child nodes mentioned above will be merged as a new tree node. In addtion, if there exists the second type of child node, this node will become the new child node of the new node whose core number is $c$+1. This process should be done for each vertex in $V^+$. We give the Algorithm~\ref{alg:moveDown} and illustrate this in Example~\ref{eg:goDown} afterwards.

\begin{algorithm}
\caption{move down vertices: {\tt moveDown}}
\label{alg:moveDown}
\footnotesize{
\algrenewcommand{\algorithmiccomment}[1]{\hskip3em$//$ #1}
\begin{algorithmic}[1]
    \Function{moveDown($V^+$, $p$)}{}
    \If{$V^+$=$\emptyset$}
        \Return $p$;
    \EndIf
    \State $P \gets \emptyset$;
    \State update $p$ using $V^+$;
    \For {each $a \in V^+$}
        \For{each $b\in a$'s neighbor vertices}
            \If{$core_G[b]>c$ and $b\notin V^+$}
                \State locate node $p_b$;
                \State run \Call{TRACE($p_b$)}{}, and update $P$;
            \EndIf
        \EndFor
    \EndFor
    \State $p_{max}\gets$ a node of $P$, which has a core number of $c$+1
        and its $vertexSet$ is the largest among all nodes of $P$;

    \If{$p_{max} = $ null }
        \State create a new node $p'$;
        \State update $p'$;
        \State add $P$ to $childList$ of $p'$;
    \Else{}
    \State add $V^+$ to $p_{max}.vertexSet$;
         \For{each $p_i \in P$}
            \If{$p_i.coreNum$ = $c+1$}
                \State merge $p_i$ to $p_{max}$;
            \Else{}
                \State  add $p_i$ to $childList$ of $p_{max}$;
            \EndIf
         \EndFor
         \State $p' \gets p_{max}$;

    \EndIf
    \State update vertex-node map;
     \If{$p.vertexSet=\emptyset$}
            \State add $ \{p.childList-P\}$ to $childList$ of $p.father$;
         \EndIf
    \State \Return $p'$;
    \EndFunction
\end{algorithmic}}
\end{algorithm}

Algorithm~\ref{alg:moveDown} presents {\tt moveDown}.
If $V^+$$\neq\emptyset$, we first initialize a node set $P$ (line 3).
Then, we remove $V^+$ from $p.vertexSet$ and update $p.invertedList$ (line 4).
$\forall$$a\in V^+$, we enumerate $a$'s neighbor $b$ whose $core_G[b]$$\textgreater$$c$,
locate $p_b$, trace up from $p_b$ to find $p_b$'s ancestor node $o_b$ which is a child node of $p$,
and put $o_b$ into $P$ (lines 5-9).
Let the node which has the largest size with core number being $c$+1 in $P$ be $p_{max}$ (line 10).
Next, if $p_{max}$=$null$, we need to create a new child node of $p$ (lines 11-14);
otherwise, we merge and reorganize $p$'s child nodes (lines 15-22).
Finally we return node $p'$ (line 26), which will be used later.

%In Algorithm~\ref{alg:moveDown}, if $V^+$ is not an empty set, we first initialize a node set $P$ and a node $p_{max}$ (line 3). After computing $V^+$, we need to remove $V^+$ from $vertexSet$ of $p$ and update $invertedList$ of $p$ (line 4). For each vertex $a\in V^+$, we locate $a$'s neighbor vertex $b$ whose core number is larger than c in the tree at the node $p_b$, then we trace up from $p_b$ with $fatherNode$ to find $p_b$'s ancestor node which is also the child node of $p$ and add it to $P$. Simultaneously, we mark the child node whose core number is $c$+1 and contains the most vertices as $p_{max}$ (lines 5-9). Next, if $p_{max}$ does not exist, a new node $p'$ will be created (line 11). Vertices in $V^+$ will be added to $vertexSet$ of $p'$ and all child nodes in $P$ will also be linked to $p'$ (lines 12-13). If $p_{max}$ exists, vertices in $V^+$ will be added into $vertexSet$ of $p_{max}$. All the first type of child nodes in $P$ will be merged to $p_{max}$, and the second type of child nodes in $P$ will be linked to $p_{max}$ (line 15-20). Then we update the vertex-node map (line 22). Note that after removing $V^+$, if $p.vertexSet$ is empty, we update the child list of $p.father$ (lines 23-24). Finally the updated node will be returned (line 25).
%Note that in the process of merging nodes, we merge nodes to the one that contains the most vertices because it is more efficient.



\textbf{Step 3: Merge subtrees.}
Recall in Figure~\ref{fig:ConnectivityExmp}, after inserting ($G$, $I$), the corresponding subtrees, which correspond to the $k$-$\widehat{core}$s containing $G$ and $I$ are merged into one subtree. The process of merging subtrees starts from the tree nodes which contain $G$ and $I$, and ends at their common ancestor node.
The merging process is guaranteed by Lemmas~\ref{lemma:mergetree} and \ref{lemma:relevantNode}.

\begin{lemma}
\label{lemma:mergetree}
After inserting an edge between two vertices, the maximum numbers of disconnected $k$-$\widehat{core}$s
which need to be merged is two.
\end{lemma}
\begin{proof}
The hypothesis is that there exist at least 3 disconnected $k$-$\widehat{core}$s that need to be merged after inserting an edge. 
We first randomly select three disconnected $k$-$\widehat{core}$s. Then we can prove by contradiction that one of them are already connected to one of the other two $k$-$\widehat{core}$s, which means there does not exist the third disconnected $k$-$\widehat{core}$ that needs to be merged. That completes the proof.  
\end{proof}

%\begin{figure}[ht]
%    \centering
%    \includegraphics[width=0.4\linewidth]{figures/MGlemma}
%    \caption{Three separate $k-\widehat{core}$.}
%    \label{fig:MGlemma}
%\end{figure}
%
%\begin{proof}
%\label{prf:proof}
%Since
% One subtree whose core number of the root is $k$, for instance, represents a $k$-$\widehat{core}$. Lemma~\ref{lemma:mergetree} means, in other words, after inserting an edge between two vertices, there are at most 2 $k$-$\widehat{core}$ that need to be merged.
%
% Figure~\ref{fig:MGlemma} presents that $G_1$, $G_2$, $G_3$ are three $k$-$\widehat{core}$, and $u \in G_1$, $v \in G_2$, $w \in G_3$ . Suppose that, after inserting an edge($u$,$v$), $G_1$ and $G_2$ is connected and need to be merged. From the definition of k$-\widehat{core}$, every vertex in $G_1$ can reach every vertex in $G_2$ along the path ($u$, $...$, $v$). If $G_3$ is also affected by the insertion which means $G_3$ needs to be merged with $G_1$ and $G_2$. There must exist one path ($w$, ..., $u$, ..., $v$). Suppose that the path is between $w$ and $v$, there are two possible paths for $w$ to reach $v$. One is directly through the path ($w$, ..., $v$), the other is through ($w$, ..., $u$) first, and then through ($u$, ..., $v$). Case one means there exists an edge between $G_2$ and $G_3$, which is contradictory to the fact that $G_2$ and $G_3$ are two separate subgraphs. Case two means there already exists one edge before the insertion, and that is also contradictory to the fact.
%
% Therefore, the number of subtrees which need to be merged is less than 3. This completes the proof.
%\end{proof}

\begin{lemma}
\label{lemma:relevantNode}
In the process of merging subtrees, the maximum number of nodes which need to be merged in each level is two.
\end{lemma}
\begin{proof}
It can be proved similarly by contradiction.
\end{proof}
%\begin{proof}
%The hypothesis that there exists a third tree node which need to be merged in each level can be proved to be false in the similar way to the proof of Lemma~\ref{prf:proof}. Thus, we do not repeat it.
%\end{proof}

By Lemmas~\ref{lemma:mergetree} and \ref{lemma:relevantNode}, we conclude that,
to merge the subtrees, we can first trace two paths starting from $p_u$ and $p_v$ until their common ancestor in the CL-tree,
and then merge the pairs of nodes on the paths, if their core numbers are the same.

Algorithm~\ref{alg:insertEdge} presents the overall steps of {\tt insertEdge}.
Following Theorem~\ref{thrm:kcoreupdate}, we first compute $V^+$,
and invoke {\tt moveDown} to update these nodes in CL-tree (lines 2-16).
Next, if $p_u'$ and $p_v'$ belong to two disconnected $k$-$\widehat{core}$s, we need to merge the subtrees (lines 17-19).
In detail, we first trace two paths starting from $p_u'$ and $p_v'$ up until one common ancestor.
Then, for each pair of nodes on the paths, if their core numbers are equal,
we merge them as a single node.
Finally, the tree index is updated.
Note that during the above process, the elements of nodes and vertex-node map are also updated.
%the efficient way is always merging the node with few vertices to the one with more vertices.

\begin{algorithm}[h]
\caption{index update algorithm: {\tt insertEdge}}
\label{alg:insertEdge}
\footnotesize{
\algrenewcommand{\algorithmiccomment}[1]{\hskip3em$//$ #1}
\begin{algorithmic}[1]
    \Function{insertEdge($p_u$,$p_v$)}{}
    \If{$p_u.coreNum$=$p_v.coreNum$}
        \State compute  $V_1^+$ in $p_u.vertexSet$;
        \State $p_u' \gets $ \Call{movedown($V_1^+$,$p_u$)}{};
        \State $p_v' \gets p_v$;
        \If {$p_u \neq p_v$}
            \State compute $V_2^+$ in $p_v.vertexSet$;
            \State $p_v' \gets $ \Call{movedown($V_2^+$,$p_v$)}{};
        \EndIf
    \ElsIf{$p_u.coreNum < p_v.coreNum$}
        \State compute $V^+$ in $p_u.vertexSet$;
        \State $p_u' \gets $ \Call{movedown($V^+$,$p_u$)}{};
        \State $p_v' \gets p_v$;
    \Else
        \State compute $V^+$ in $p_v.vertexSet$;
        \State $p_v' \gets $ \Call{movedown($V^+$,$p_v$)}{};
        \State $p_u' \gets p_u$;
    \EndIf
     \If{$p_u'$ and $p_v'$ are in two disconnected $k$-$\widehat{core}$s}
        \State trace two paths starting from $p_u'$ and $p_v'$ up until a common ancestor;
        \State merge pairs of nodes with the same core number on the paths;
    \EndIf
    \EndFunction

\end{algorithmic}}
\end{algorithm}


\subsection{Edge Deletion}
\label{sec:edgeDeletion}

Similar to the edge insertion, deleting an edge may trigger the updates of CL-tree nodes as well as their structures. We illustrate this by Example~\ref{ep:deleteExample}.

\begin{example}
\label{ep:deleteExample}
Consider the graph in Figure~\ref{fig:connectivity}. If we delete an edge ($H$, $G$) of the graph in Figure~\ref{fig:advancedIndex}, the core number of vertex $H$ decreases to 1. Thus we need to create a new node with core number being 1 and then move $H$ up to the new node. If we delete an edge ($G$, $I$), the connectivity of some vertices changes as shown in Figure~\ref{fig:advancedIndex} and thus the corresponding subtree has to be split to two new ones.
\end{example}
To maintain the CL-tree for deleting an edge, we propose an algorithm called {\tt deleteEdge}.
Let $V^-$ be the set of vertices whose core numbers decrease after deleting an edge ($u$, $v$).
We summarize the main steps of {\tt deleteEdge} as follows.

$\bullet$ \textbf{Step 1:} Compute $V^-$;

$\bullet$ \textbf{Step 2:} Split the subtree;

$\bullet$ \textbf{Step 3:} Move up vertices of $V^-$.

We now elaborate these steps one by one.

\textbf{Step 1: Compute $V^-$.}
By Lemma~\ref{lm:kcorelemma}, the core numbers of vertices in $G$ decrease by at most 1 after deleting an edge.
We compute $V^-$ using the algorithm in~\cite{kcoreUpdate}.

\textbf{Step 2: Split the subtree.}
Similar to edge insertion, the connectivity of vertices may change after deleting an edge. We illustrate this by Example~\ref{ep:delete}. 

\begin{example}
\label{ep:delete}
Figure~\ref{fig:splitExmp} presents two 2-$\widehat{core}$s. After deleting an edge ($u,v$), the graph in Figure~\ref{fig:split1} is still a 2-$\widehat{core}$; however, the graph in Figure~\ref{fig:split2} is split to two disconnected 2-$\widehat{core}$s.
\end{example}

\begin{figure}[ht]
    \centering
    \mbox{
        \subfigure[]{
            \includegraphics[width=.335\columnwidth]{figures/splitExmp1}
            \label{fig:split1}
        }
        \hspace{2ex}
        \subfigure[]{
            \includegraphics[width=.335\columnwidth]{figures/splitExmp2}
            \label{fig:split2}
        }
    }
    \caption{The connectivity change.}
    \label{fig:splitExmp}
\end{figure}

Inspired by Example~\ref{ep:delete}, we need to reorganize the subgraph and determine whether the subtree will be split after deleting the edge ($u, v$). Instead of rebuilding the subtree, here we dynamically update the subtree in a bottom-up manner starting from the node $p$ ($p.getCore$ = min\{$core_G[u], core_G[v]$\}). We first introduce a vertex-tree map. The key of this map is a vertex, and corresponding value is the root node of one subtree. We illustrate the map in Example~\ref{em:reorganize}. 


%Moreover, if this node is split to two nodes, the father node of this node is also possible to be split because of the new child node. Thus we need to split the subtree level by level until no node needs to be split.
%Let the tree node be $p$ which $u$ and $v$ belong to.
%We first give an observation: If $p$ is a leaf node, vertices in $p$ form a $k$-$\widehat{core}$. If $p$ is a non-leaf node, vertices in $p$ are not necessarily connected from each other. See Figure~\ref{fig:cktree}, vertices $F, G$ in $r_1$ are not directly connected because the CL-tree is compressed.
%Based on the observation, we summarize that vertices which form a tree node should satisfy the following conditions:

%$\bullet$ Vertices in a connected component.

%$\bullet$ Vertices share the same child node.

\begin{figure}[ht]
    \centering
    \includegraphics[width=0.8\linewidth]{figures/traceChild}
    \caption{The vertex-tree map.}
    \label{fig:trace}
\end{figure}


\begin{example}
\label{em:reorganize}
Consider the subtree shown in Figure~\ref{fig:trace}, let $x,y$ denote two vertices of the node $p$, and $p_1,p_2$ are two child nodes of $p$. Vertex $w$ and $z$ are neighbors of $x$ and $y$ respectively and their core numbers are both larger than $x$'s and $y$'s. For $w$, we first locate $p_w$ that $w$ belongs to and trace up to find $p_1$ which is the ancestor node of $p_w$ as well as one child node of $p$. Thus the value of $w$ in this map is $p_1$. Similarly, $p_2$ is mapped to $z$.
\end{example}
After traversing the subtree and building the map, we can now reorganize vertices in $p$ by enumerating their neighbors. For instance, $p_1$ is now associated with $x$ because $x$ has a neighbor $w$. If $p_1$ is associate with another vertex in $p$, then we can group this vertex with $x$ in one vertex set because they are connected through some vertices of the subtree whose root is $p_1$. Note that if we find one of $x$'s neighbor vertices whose core number is equal to $x$'s, we can directly group it with $x$ because this vertex is definitely being in $p$ as well. 
After reorganizing vertices in $p$, we can determine whether the node should be split. Note that the maximum number of nodes that $p$ may be split to is 2 (recall Lemma~\ref{lemma:relevantNode}), if vertices in $p$ are eventually grouped in two sets, then we split this node and repeat the process in $p.father$. 
The terminating condition of spliting the subtree is all the vertices of $p$ are grouped in one set and all the child nodes of $p$ are shared by vertices in this set. We give Example~\ref{em:steps} to illustrate the process.

\begin{figure}[ht]
    \centering
    \includegraphics[width=1\linewidth]{figures/steps}
    \caption{The process of spliting the subtree.}
    \label{fig:steps}
\end{figure}

\begin{example}
\label{em:steps}
Consider the original graph in Figure~\ref{fig:split2}, the process of spliting the subtree is shown in Figure~\ref{fig:steps}. After delete the edge ($G,I$), we first locate the node $p_3$ contains $G$, reorganize vertices of $p_3$ and find that $p_2$ is associate with no vertex of $p_3$. Although $p_3$ remains complete, we continue the process in $P_4$. Then, after reorganizing vertices in $p_4$, we split it to two nodes because $p_2$ and $p_3$ are respectively shared by vertices $H$ and $M$. Next, $p_5$ remains unaffected and all its child nodes are shared by $N$ in $p_5$. Finally we finish spliting the subtree.  
\end{example}

%If $p$ is split to 2 nodes then we continue the process in $p.father$ until no node needs to be split.
%Note that we separate the node from the one whose has more vertices because it is more efficient.



\textbf{Step 3: Move up vertices of $V^-$.}

After computing the vertex set $V^-$ and modifying the structure of the tree index, the third step of our edge deletion algorithm is moving up vertices of $V^-$ if necessary. We give the {\tt moveUp} algorithm in Algorithm~\ref{alg:moveUp}.

As outlined in Algorithm~\ref{alg:moveUp}, we first initialize two node sets $P,P'$(line 3). We need to remove $V^-$ from $vertexSet$ of $p$ and update $invertedList$ of $p$ (line 4). Then if the core number of $father$ is $c-1$, we add $V^-$ to it; If the core number of $father$ is less than $c-1$, we create a new node and join it to the tree(lines 5-10). Next, we collect the left vertices of $p$ in $set$, reorganize and split them to nodes and update the vertex-node map (lines 11-13). We also need to update $childList$ and $invertedList$ of $p.father$ (lines 15-16). If there exist child nodes that are not visited in line 12, we re-link them to $p.father$ because these nodes are traceable only by vertices of $V^-$ (lines 16-18).


\begin{algorithm}[h]
\caption{move up vertices: {\tt moveUp}}
\label{alg:moveUp}
\footnotesize{
\algrenewcommand{\algorithmiccomment}[1]{\hskip3em$//$ #1}
\begin{algorithmic}[1]
\Function{moveUp($V^-$, $p$)}{}
\If{$V^- \neq \emptyset$}
    \State $P,P'\gets \emptyset$;
    \State update $p$;
    \If{ $ (p.father).getCore$ = $c$-1}
    %\If{$father.getCore = c-1$}
       \State add $V^-$ to $father$;
    \Else
       \State create new node $newFather$;
       \State add $V^-$ to $newFather$;
       \State join $newFather$ to the tree;
    \EndIf
  \State $set \gets$ $p.vertexSet$;
  \State $P \gets$ reorganize vertices of $set$ and split to nodes;
  \State update vertex-node map;
  \State link each $p_i \in P$ to $p.father$;
  \State update $invertedList$ of $p.father$;

  \State $P' \gets$ get child nodes which are not visited by the re-oranize step;
  \If{$P' \neq \emptyset$}
    \State re-link each $p \in P'$ to $p.father$;
  \EndIf
\EndIf
\EndFunction
\end{algorithmic}}
\end{algorithm}


We outline the edge deletion algorithm in Algorithm~\ref{alg:delete}. Similar to edge insertion, we have three cases to handle separately. In these three cases, we first compute vertex set $V^-$ (lines 3,8,13). Then we split the subtree (lines 4,9,14). Next we apply {\tt moveUp} to update vertices of $V^-$ (lines 6,11,17). In $p_u=p_v$ case, if the tree is split to 2 parts, we should separate $V^-$ to two sets and invoke {\tt moveUp} accordingly (lines 19-21).

\begin{algorithm}[h]
\caption{index algorithm: {\tt deleteEdge}}
\label{alg:delete}
\footnotesize{
\algrenewcommand{\algorithmiccomment}[1]{\hskip3em$//$ #1}
\begin{algorithmic}[1]
\Function{deleteEdge($p_u,p_v$)}{}
    \If{$p_u.coreNum > p_v.coreNum$}
        \State compute $V^-$ in $p_v.vertexSet$;
        \State split the subtree;
        \State locate node $p_v'$;
        \State \Call{moveUp($V^-$,$p_v'$)}{};
    \ElsIf{$p_u.coreNum < p_v.coreNum$}
        \State compute $V^-$ in $p_u.vertexSet$;
        \State split the subtree;
        \State locate node $p_u'$;
        \State \Call{moveUp($V^-$,$p_u'$)}{};
    \Else
        \State compute $V^-$ in $p_u.vertexSet$;
        \State split the subtree;
        \State locate node $p_u',p_v'$;
        \If{$p_u' = p_v'$}
             \State \Call{moveUp($V^-$,$p_u'$)}{};
        \Else
             \State ${V_u}^-, {V_v}^- \gets$ separate $V^-$;
             \State \Call{moveUp(${V_u}^-$,$p_u'$)}{};
             \State \Call{moveUp(${V_v}^-$,$p_v'$)}{};
        \EndIf

    \EndIf
\EndFunction
\end{algorithmic}}
\end{algorithm}




} 