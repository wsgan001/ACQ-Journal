\section{Variants of LAC Search}
\label{variant}
As the context cohesiveness of LACs can be defined in various ways,
there are many possible variants of LAC search.
In this paper, we consider \chen{three} typical variants.

\begin{variant}
\label{variant1}
Given a graph $G(V,E)$, an integer $k$, a query vertex $q\in V$ and a predefined label $S$,
return the LAC, which satisfies the first two constraints of Problem~\ref{problem1}
and its label is exactly $S$.
\end{variant}


\begin{variant}
\label{variant2}
Given a graph $G(V,E)$, an integer $k$, a query vertex $q\in V$,
a predefined label $S$ and a threshold $\theta (0\leq \theta\leq 100)$,
return the LAC, which satisfies the first two constraints of Problem~\ref{problem1}
and every community member shares at least $\theta$ percentage of keywords in $S$.
\end{variant}

Variant~\ref{variant1} can be applied to applications,
which need communities having specific keyword constraints.
Note that $S$ does not necessary to be the subset of $W(q)$.
In Variant~\ref{variant2}, the context cohesiveness is relaxed.
This can be applied for cases when the context information is weak.
We illustrate the variants in Example~\ref{eg:variant}.

\begin{example}
\label{eg:variant}
Consider Figure~\ref{fig:kcoreGraph}. Let $q$ be $A$ and $k$ be 2.
For Variant~\ref{variant1}, if the predefined label is $\{$x$\}$,
then the vertex set $\{A,B,C,D\}$ forms a target LAC.
For Variant~\ref{variant2}, if the predefined label is $\{$x,y$\}$
and the threshold is $50\%$,
then the vertex set $\{A,B,C,D,E\}$ forms a target LAC.
\end{example}

We first introduce the solutions to Variant~\ref{variant1}.
In line with Problem~\ref{problem1}, we first introduce the basic solutions without index,
which are extended naturally from {\tt basic-g} and {\tt basic-w},
and are denoted by {\tt basic-g-v1} and {\tt basic-w-v1} respectively.
Their details are presented in Appendix~\ref{app:variant}.
For the index based algorithm, we denote it by {\tt SW} (search by keywords).
The main steps are:
(1) finding node $r_k$, whose corresponding $k$-$\widehat {core}$ contains $q$, from CL-tree;
(2) collecting vertices containing $S$ by traversing the subtree rooted at $r_k$ and intersecting the inverted lists;
(3) finding $G[S,q]$ from the subgraph induced by these vertices;
(4) finding $G_k[S,q]$ from $G[S,q]$;
and (4) output $G_k[S,q]$ as the target LAC.
The solutions of Variant~\ref{variant2} can be adapted from all the query algorithms of Variant~\ref{variant1}.
We denote them by {\tt basic-g-v2}, {\tt basic-w-v2} and {\tt SWT} respectively.
We present the pseudocodes of solutions to Variant~\ref{variant1}
and needed adaption for Variant~\ref{variant2} in Appendix~\ref{app:variant}.

In addition, it would be interesting to study the combinations of
Problem~\ref{problem1} with Variants~\ref{variant1} and~\ref{variant2}.
The solutions of Problem~\ref{problem1} and Variants~\ref{variant1} and~\ref{variant2}
can be easily extended to solves these combinations.
We skip the details due to the space limitation. 