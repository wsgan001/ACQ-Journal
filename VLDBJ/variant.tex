{\color{blue}
\section{Variants of ACQ Problem}
\label{variant}

In this section, we introduce two typical variants and present the query algorithms based on the CL-tree.


\subsection{Variant 1}

We first present an approximation version of the ACQ query, denoted by Variant~\ref{df:variant1}.
In Variant~\ref{df:variant1}, vertices of an AC do not need to exactly share the same keywords in $S$;
instead, they just need to share a predefined percentage of keywords in $S$.
Thus, the keyword cohesiveness is relaxed.
This could be useful for graphs if the keyword information of vertices is weak.

\begin{variant}
\label{df:variant1}
Given a graph $G$, a positive integer $k$, a vertex $q \in V$, a predefined keyword set $S$,
and a threshold $\theta\in$[0,1], return a subgraph $G_q$, the following properties hold:

$\bullet$ \textbf{Connectivity}. $G_q \subseteq G$ is connected and $q\in G_q$;

$\bullet$ \textbf{Structure cohesiveness}. $\forall$$v\in G_q$, $deg_{G_q}(v)\geq$$k$;

$\bullet$ \textbf{Keyword cohesiveness}. $\forall v\in G_q$, it has at least $|S|\times \theta$ keywords in $S$.
\end{variant}

We illustrate Variant~\ref{df:variant1} using Example~\ref{eg:variant1}.
\begin{example}
\label{eg:variant1}
In Figure~\ref{fig:kcoreGraph}, let $q$=$A$ and $k$=2.
If $S$=$\{x,y\}$, $\theta$=50\%, Variant~\ref{df:variant1} will return the subgraph induced by the vertex set $\{A,B,C,D,E\}$ as the target AC.
\end{example}

In line with Problem~\ref{problem1}, we first introduce the basic solutions without index,
which are extended naturally from {\tt basic-g} and {\tt basic-w},
and are denoted by {\tt basic-g-v1} and {\tt basic-w-v1} respectively.
Their detailed algorithms are presented in Appendix~\ref{app:algoOfVariant}.

We also propose an efficient query algorithm {\tt SWT}, based on the CL-tree index.
Algorithm~\ref{alg:swt} presents {\tt SWT}. We first apply {\tt core-locating} to find node $r_k$, whose corresponding $k$-$\widehat {core}$ contains $q$, from CL-tree (line 1).
Then we traverse the subtree rooted at $r_k$, and collect a set $V'$ of vertices containing at least $|S|\times \theta$ keywords by applying {\tt keyword-checking}.
Next, we find $G[S]$ from the subgraph induced by vertices in $V'$ (line 3),
and find $G_k[S]$ from $G[S]$ (line 4).
Finally, we output $G_k[S]$ as the target AC, if it exists (line 5).

\begin{algorithm}[h]
\caption{Query algorithm: {\tt SWT}}
\label{alg:swt}
\footnotesize{
\algrenewcommand{\algorithmiccomment}[1]{\hskip3em$//$ #1}
\begin{algorithmic}[1]
    \Function{query($G$, $root$, $q$, $k$, $S$)}{}
        \State find the node $r_k$ from the CL-tree index;
        \State traverse the subtree rooted at $r_k$ and
               collect a set $V'$ of vertices containing at least $|S|\times \theta$ keywords by intersecting the inverted lists;
        \State find $G[S]$ from the subgraph induced by $V'$;
        \State find $G_k[S]$ from $G[S]$;
        \State output $G_k[S]$ as the target AC.
    \EndFunction
\end{algorithmic}}
\end{algorithm}


\subsection{Variant 2}

To extend the ACQ query for supporting mutiple query vertices, we propose Variant~\ref{df:variant2} as follows.

\begin{variant}
\label{df:variant2}
Given a graph $G$, a positive integer $k$, a vertex set $Q$$\subseteq$$V$, and a predefined keyword set $S$, return a subgraph $G_Q$, the following properties hold:

$\bullet$ \textbf{Connectivity}. $G_Q \subseteq G$ is connected and $G_Q$ contains all the vertices of $Q$;

$\bullet$ \textbf{Structure cohesiveness}. $\forall$$v\in G_Q$, $deg_{G_Q}(v)\geq$$k$;

$\bullet$ \textbf{Keyword cohesiveness}. The size of $L(G_Q, S)$ is maximal, where $L(G_Q, S)=\cap_{v \in G_Q}(W(v)\cap S)$ is the set of keywords shared in $S$ by all vertices of $G_Q$.
\end{variant}

We illustrate Variant~\ref{df:variant2} via Example~\ref{eg:variant2}.

\begin{example}
\label{eg:variant2}
In Figure~\ref{fig:kcoreGraph}, let $Q$=$\{A,C\}$ and $k$=2.
If $S$=$\{w,x,y,z\}$, then Variant~\ref{df:variant2} returns the subgraph induced by the vertex set $\{A,C,D\}$ as the target AC, whose shared keyword set is$\{x,y\}$.
\end{example}

To answer the query in Variant~\ref{df:variant2}, we can first find a set $S'$ of intersected keywords, which are contained by $S$ and every vertex in $Q$. Then, we randomly take a vertex $q$$\in$$Q$ as the query vertex.
Finally, we can find the target ACs by any of previous ACQ algorithms.
Following the above idea, we extend {\tt basic-g} and {\tt basic-w} and obtain two basic algorithms, i.e., {\tt basic-g-v2} and {\tt basic-w-v2}. We also extend {\tt Dec} and get an index based algorithm {\tt MDec}.
Algorithm~\ref{alg:MDec} presents {\tt MDec}.
Note that we do not extend {Inc-S} and {Inc-T}, as they are generally slower than {\tt Dec}, which has been demonstrated by the early version of this paper~\cite{fangeffective}.

\begin{algorithm}[htp]
\caption{Query algorithm: {\tt MDec}}
\label{alg:MDec}
\footnotesize{
\algrenewcommand{\algorithmiccomment}[1]{\hskip3em$//$ #1}
\begin{algorithmic}[1]
    \Function{query($G$, $root$, $Q$, $k$, $S$)}{}
      \State $S'$ = $( \bigcap_{i=0}^{|Q|-1}W(q_i))\cap S$;
      \State $q \gets$ randomly select a vertex from $Q$;
      \State run {\tt Dec} with $q$, $k$, and $S'$;
      \State output target ACs which contain $Q$;
    \EndFunction
\end{algorithmic}}
\end{algorithm}

} 