\section{Details of Basic Solutions}
\label{app:basic}

Algorithms~\ref{alg:basic-g} and~\ref{alg:basic-w} present {\tt basic-g} and {\tt basic-w} respectively.
The input of {\tt basic-g} is a graph $G$, a query vertex $q$, an integer $k$ and a set $S$.
It first generates a set, $\Psi$, of candidate keyword sets,
each of which contains a single keyword of $S$ (line 2).
Then, it finds the $k$-$\widehat {core}$, ${\mathcal C}_k$, containing $q$ from the graph $G$.
In the while loop (lines 4-14), it first initializes an empty set $\Phi$ (line 5),
which is used to collect all the qualified keyword sets.
Then for each candidate keyword set $S'\in \Psi$,
it finds $G[S']$ from ${\mathcal C}_k$ by considering the keyword constraint.
After that, it finds $G_k[S']$ from $G[S']$ (lines 7-8),
and put it into $\Phi$ if $G_k[S']$ exists (lines 9-10).
After checking all the candidate keyword sets in $\Psi$,
if there are at least one qualified keyword sets in $\Phi$,
it generates a new set $\Psi$ of larger candidate keyword sets
by calling \textsc{geneCand($\Phi$)} (see Appendix~\ref{app:geneCand})
and continues to checking longer candidate keyword sets in next loop;
otherwise, it stops and outputs all the communities of the latest verified keyword sets as the target ACs.


\begin{algorithm}[h]
\caption{Basic solution: {\tt basic-g}}
\label{alg:basic-g}
\footnotesize{
\algrenewcommand{\algorithmiccomment}[1]{\hskip3em$//$ #1}
\begin{algorithmic}[1]
    \Function{query($G$, $q$, $k$, $S$)}{}
        \State init $\Psi$ using $S$;
        \State find the $k$-$\widehat {core}$, ${\mathcal C}_k$, containing $q$ from $G$;
        \While {true}
            \State $\Phi\gets \emptyset$;
            \For {each $S'$ $\in \Psi$}
                \State find $G[S']$ from ${\mathcal C}_k$;
                \State find $G_k[S']$ from $G[S']$;
                \If {$G_k[S']$ exists}
                    \State $\Phi$.add($S'$);
                \EndIf
            \EndFor
            \If {$\Phi \ne \emptyset$}
                \State $\Psi\gets$ \Call{geneCand($\Phi$)}{};
            \Else
                \State break;
            \EndIf
        \EndWhile
        \State output the communities of keyword sets in $\Phi$;
    \EndFunction
\end{algorithmic}}
\end{algorithm}

Algorithm~\ref{alg:basic-w} presents the pseudocodes of {\tt basic-w}.
It follows the main steps of {\tt basic-g},
except that for each candidate keyword set $S'$,
it finds $G[S']$ from $G$ directly,
rather than from ${\mathcal C}_k$.

\begin{algorithm}[h]
\caption{Basic solution: {\tt basic-w}}
\label{alg:basic-w}
\footnotesize{
\algrenewcommand{\algorithmiccomment}[1]{\hskip3em$//$ #1}
\begin{algorithmic}[1]
    \Function{query($G$, $q$, $k$, $S$)}{}
        \State init $\Psi$ using $S$;
        \While {true}
            \State $\Phi\gets \emptyset$;
            \For {each $S'$ $\in \Psi$}
                \State find $G[S']$ from $G$;
                \State find $G_k[S']$ from $G[S']$;
                \If {$G_k[S']$ exists}
                    \State $\Phi$.add($S'$);
                \EndIf
            \EndFor
            \If {$\Phi \ne \emptyset$}
                \State $\Psi\gets$ \Call{geneCand($\Phi$)}{};
            \Else
                \State break;
            \EndIf
        \EndWhile
        \State output the communities of keyword sets in $\Phi$;
    \EndFunction
\end{algorithmic}}
\end{algorithm}
