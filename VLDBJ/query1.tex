\subsection{The Incremental Algorithms}
\label{inc}

While the high-level idea of incremental algorithms resembles the basic solutions (see Section~\ref{basic}),
{\tt Inc-S} and {\tt Inc-T} advance them with the exploitation of the CL-tree.
Specifically, they can always verify the existence of $G_k[S']$ within a subgraph of $G$ instead of the entire graph $G$. More interestingly, the subgraph for such verifications shrinks when the candidate set $S'$ expands. Therefore, a large sum of redundant computation is cut off during the verification. We present {\tt Inc-S} and {\tt Inc-T} in Sections~\ref{inc-S} and~\ref{inc-T}.


\subsubsection{Inc-S Algorithm}
\label{inc-S}
We first introduce a new concept, called \textbf{subgraph core number},
which is geared to the main idea of {\tt Inc-S}.

\begin{definition}[Subgraph core number]
\label{def:ccscore}
  The core number of a subgraph $G'$ of $G$, $core_G[G']$,
  is defined as $min\{core_G[v]|$ $v\in G'\}$.
\end{definition}


{\tt Inc-S} follows the two-step framework (\emph{verification} and \emph{candidate generation})
introduced in Section~\ref{basic}. With the CL-tree, we improve the verification step as follows.
\begin{itemize}
\item {\bf Core-based verification.} For each newly generated size-($c$+1) candidate keyword set $S'$ expanded from size-$c$ sets $S_1$ and $S_2$, mark $S'$ as a qualified set if $G_k[S']$ exists \textit{in a subgraph of core number} $max\{core_G[G_k[S_1]]$, $core_G[G_k[S_2]]\}$.
\end{itemize}

The core-based verification guarantees that, with the expansion of the candidate keyword sets, the verification becomes faster as it only needs to examine the existence of $G_k[S']$ in a smaller $k$-$\widehat {core}$ (Recall that cores with large core numbers are nested in the cores with small core numbers). The correctness of such shrunk verification range is guaranteed by the following lemma.
\begin{lemma}
\label{lemma:coreDown}
  Given two subgraphs $G_k[S_1]$ and $G_k[S_2]$ of a graph $G$,
  for a new keyword set $S'$ generated from $S_1$ and $S_2$ (\textit{i.e.}, $S'=S_1\cup S_2$),
  if $G_k[S']$ exists, then it must appear in a $k$-$\widehat {core}$ with core number at least
  \begin{equation}
    max\{core_G[G_k[S_1]], core_G[G_k[S_2]]\}.
  \end{equation}
\end{lemma}

The verification process can be further accelerated by checking the numbers of vertices and edges,
as indicated by Lemma~\ref{lemma:coreExist}.
\begin{lemma}
\label{lemma:coreExist}
  Given a connected graph $G(V,E)$ with $n$=$|V|$ and $m$=$|E|$,
  if $m - n < \frac{{{k^2} - k}}{2} - 1$, there is no $k$-$\widehat {core}$ in $G$.
\end{lemma}

This lemma implies that, for a connected subgraph $G'$, whose edge and vertex numbers are $m$ and $n$,
if $m - n < \frac{{{k^2} - k}}{2} - 1$, then we cannot find $G_k[S']$ from $G'$.

We present {\tt Inc-S} in Algorithm~\ref{alg:incS}.
The input is a CL-tree rooted at $root$, a query vertex $q$, a positive integer $k$ and a keyword set $S$.
We apply {\tt core-locating} on the CL-tree to locate the internal nodes whose corresponding $k$-$\widehat {core}$s contain $q$ (line 2).
Note that their core numbers are in the range of $[k,core_G[q]]$, as required by the structure cohesiveness.
Then, we set $l$=0, indicating the sizes of current keyword sets, and initialize a set $\Psi$ of $\textless S',c\textgreater$ pairs,
where $S'$ is a set containing a keyword from $S$ and $c$ is the initial core number $k$ (line 3).
Note that we skip those keywords, which are in $S$, but not in $W(q)$.
In the while loop (lines 4-18), for each $\textless S',c\textgreater$ pair,
we first perform {\tt keyword-checking} to find $G[S']$ using the keyword inverted lists of the subtree rooted at node $r_c$.
If we cannot ensure that $G[S']$ does not contain a $k$-$\widehat {core}$ by Lemma~\ref{lemma:coreExist},
we then find $G_k[S']$ from $G[S']$ (lines 8-9).
If $G_k[S']$ exists, we put $S'$ with its core number into the set $\Phi_l$ (lines 10-11).
Next, if $\Phi_l$ is nonempty, we generate new candidates by calling \textsc{geneCand($\Phi_{l}$)},
which is detailed in Appendix~\ref{app:geneCand}.
For each candidate $S'$ in $\Psi$, we compute the core number using Lemma~\ref{lemma:coreDown}
and update it as a pair in $\Psi$ (lines 12-17);
otherwise, we stop (line 18).
Finally, we output the communities of the latest verified keyword sets (line 19).

\begin{algorithm}[h]
\caption{Query algorithm: {\tt Inc-S}}
\label{alg:incS}
\footnotesize{
\algrenewcommand{\algorithmiccomment}[1]{\hskip3em$//$ #1}
\begin{algorithmic}[1]
    \Function{query($G$, $root$, $q$, $k$, $S$)}{}
        \State find subtree root nodes $r_k,r_{k+1},\cdots,r_{core_G[q]}$;
        \State initialize $l$=0, $\Psi$ using $S$;
        \While{$true$}
            \State $l\gets l+1$; $\Phi_{l} \gets \emptyset$;
            \For {each $\textless S',c \textgreater$ $\in \Psi$}
                \State find $G[S']$ under the root $r_c$;
                \If {$G[S']$ is not pruned by Lemma~\ref{lemma:coreExist}}
                    \State find $G_k[S']$ from $G[S']$;
                    \If {$G_k[S']$ exists}
                        \State $\Phi_{l}$.add($\textless S', core_G[G_k[S']]\textgreater$);
                    \EndIf
                \EndIf
            \EndFor

            \If{$\Phi_{l} \ne \emptyset$}
                \State $\Psi \gets$ \Call{geneCand($\Phi_{l}$)}{};
                \For {each $S'$ in $\Psi$}
                    \If {$S'$ is generated from $S_1$ and $S_2$}
                        \State $c\gets max\{core_G[G_k[S_1]],core_G[G_k[S_2]]\}$;
                        \State $\Psi$.update($S'$, $\textless S',c\textgreater$);
                    \EndIf
                \EndFor
            \Else {} break;
            \EndIf
        \EndWhile
        \State output the communities of keyword sets in $\Phi_{l-1}$;
    \EndFunction
\end{algorithmic}}
\end{algorithm}

\begin{example}
\label{eg:inc-S}
Consider the graph in Figure~\ref{fig:kcoreGraph}
and its index in Figure~\ref{fig:cktree}.
Let $q$=$A$, $k$=1 and $S$=$\{w,x,y\}$.
By Algorithm~\ref{alg:incS}, we first find 3 root nodes $r_1$, $r_2$ and $r_3$.
In the first while loop, we find 2 qualified keyword sets $\{x\}and \{y\}$ with core numbers being 3 and 1.
By Lemma~\ref{lemma:coreDown}, we only need to verify the new candidate keyword set $\{x,y\}$ under node $r_3$.
\end{example}

\subsubsection{Inc-T Algorithm}
\label{inc-T}
We begin with a lemma which inspires the design of {\tt Inc-T}.
\begin{lemma}
\label{lemma:kcoreIntersect}
  Given two keyword sets $S_1$ and $S_2$, if $G_k[S_1]$ and $G_k[S_2]$ exist, we have
  \begin{equation}
    %{G_k}[{S_1} \cup {S_2},q] \subseteq {G_k}[{S_1},q] \cap {G_k}[{S_2},q].
    G_k[S_1\cup S_2] \subseteq G_k[S_1]\cap G_k[S_2].
  \end{equation}
\end{lemma}

This lemma implies, if $S'$ is generated from $S_1$ and $S_2$,
we can find $G_k[S']$ from $G_k[S_1]\cap G_k[S_2]$ directly.
Since every vertex in $G_k[S_1]\cap G_k[S_2]$ contains both $S_1$ and $S_2$,
we do not need to consider the keyword constraint again when finding $G_k[S']$.

Based on Lemma~\ref{lemma:kcoreIntersect}, we introduce a new algorithm \textbf{{\tt Inc-T}}. Different from {\tt Inc-S}, {\tt Inc-T} maintains $G_k[S']$ rather than $core_G[$ $G_k[S']]$ for each qualified keyword set $S'$. As we will demonstrate later, {\tt Inc-T} is more effective for shrinking the subgraphs containing the ACs, and thus more efficient. As a trade-off for better efficiency, {\tt Inc-T} consumes more memory as it needs to store a list of subgraph $G_k[S']$ in memory.
%The detailed algorithm pseudocodes are listed in the technical report~\cite{TR}.

Algorithm~\ref{alg:incT} presents the steps of {\tt Inc-T}.
We first apply {\tt core-locating} to find the $k$-$\widehat{core}$ containing $q$ from the CL-tree (line 2).
Then, we set $l$=0, indicating the sizes of current keyword sets,
and initialize a set $\Psi$ of $<S',\widehat G>$ pairs,
where $S'$ is a set containing a keyword from $S$ and $\widehat G$ is the $k$-$\widehat {core}$.
The while loop (lines 4-18) is similar with that of {\tt Inc-S}.
The main differences are that:
(1) for each qualified keyword set $S'$, {\tt Inc-T} keeps $G_k[S']$ in memory (line 11);
and (2) for each candidate keyword set $S'$ generated from $S_1$ and $S_2$,
{\tt Inc-T} finds $G_k[S']$ from $G_k[S_1]\cap G_k[S_2]$ directly without further keyword verification (lines 6-9, 16).

\begin{algorithm}[h]
\caption{Query algorithm: {\tt Inc-T}}
\label{alg:incT}
\footnotesize{
\algrenewcommand{\algorithmiccomment}[1]{\hskip3em$//$ #1}
\begin{algorithmic}[1]
    \Function{query($G$, $root$, $q$, $k$, $S$)}{}
        \State find the $k$-$\widehat {core}$, which contains $q$;
        \State initialize $l$=0, $\Psi$ using $S$;
        \While{$true$}
            \State $l\gets l+1$; $\Phi_{l} \gets \emptyset$;
            \For {each $<S',\widehat{G}>$ $\in \Psi$}
                \State find $G[S']$ from $\widehat{G}$;
                \If {$G[S']$ is not pruned by Lemma~\ref{lemma:coreExist}}
                    \State find $G_k[S']$ from $G[S']$;
                    \If {$G_k[S']$ exists}
                        \State $\Phi_{l}$.add($<S', G_k[S']>$);
                    \EndIf
                \EndIf
            \EndFor
            \If {$\Phi_{l} \ne \emptyset$}
                \State $\Psi \gets$ \Call{geneCand($\Phi_{l}$)}{};
                \For {each $S' \in \Psi$}
                    \If {$S'$ is generated from $S_1$ and $S_2$}
                        \State $\widehat G \gets G_k[S_1]\cap G_k[S_2]$;
                        \State $\Psi_{l}$.update($S'$,$<S',\widehat G>$);
                   \EndIf
                \EndFor
            \Else {} break;
            \EndIf
        \EndWhile
        \State output the communities of keyword sets in $\Phi_{l-1}$;
    \EndFunction
\end{algorithmic}}
\end{algorithm}

\begin{example}
\label{eg:inc-T}
Continue the graph and query ($q$=$A$, $k$=1, $S$=$\{w$, $x,y\}$) in Example~\ref{eg:inc-S}.
By {\tt Inc-T}, we first find $G_1[\{x\}]$ and $G_1[\{y\}]$,
whose vertex sets are $\{A,B,C,D\}$ and $\{A,C,D$, $E,F,G\}$.
Then to find $G_1[\{x,y\}]$, we only need to search it from the subgraph,
induced by the vertex set $\{A,C,D\}$.
\end{example}



