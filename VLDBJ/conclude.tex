\section{Conclusions}
\label{conclusion}

An AC is a community that exhibits structure and keyword cohesiveness. To facilitate ACQ evaluation, we develop the CL-tree index and its query algorithms. We further propose index maintenance algorithms for dynamic graphs and define two typical variants of the ACQ problems.
Our experimental results show that ACs are
easier to interpret than those of existing community detection/search methods,
and they can be ``personalized''. Our solutions are also faster than existing community search algorithms.

We will study the use of other measures of structure cohesiveness (e.g., $k$-truss, $k$-clique) and keyword cohesiveness (e.g., Jaccard similarity and string edit distance) in the ACQ definition.
We will also investigate how the directions of edges will affect the formation of an AC.
We will examine how graph pattern matching techniques~\cite{GPM-KDD2007,GPM-VLDB2010,GPM-PVLDB2015} can be extended to find ACs. An interesting research direction is to study how to automatically generate a meaningful graph pattern that reflects a real community, and how to use these patterns to find ACs.

% Old (Before Jun 5)
%In this paper, we examine the ACQ problem, which finds communities that exhibit both structure and keyword cohesiveness. To enable efficient ACQ, we develop the CL-tree index and its query algorithms.
%Our experimental results show that ACs are easier to interpret than those of existing community detection/search methods,
%and they can be ``personalized''. Our solutions are also faster than existing community search algorithms.
%In the future, we will investigate keyword cohesiveness in other community definitions (e.g., $k$-truss and $k$-clique). 