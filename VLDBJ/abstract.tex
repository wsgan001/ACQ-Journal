Given a graph $G$ and a vertex $q \in G$, the {\it community search} query returns a subgraph of $G$ that contains vertices related to $q$. Communities, which are prevalent in {\it attributed graphs} such as social networks and knowledge bases, can be used in emerging applications such as product advertisement and setting up of social events.
In this paper, we investigate the {\it attributed community query} (or ACQ), which returns an {\it attributed community} (AC) for an {\it attributed graph}. The AC is a subgraph of $G$, which satisfies both {\it structure cohesiveness} (i.e., its vertices are tightly connected) and {\it keyword cohesiveness} (i.e., its vertices share common keywords).  The AC enables a better understanding of how and why a community is formed (e.g., members of an AC have a common interest in music, because they all have the same keyword ``music'').  An AC can be ``personalized''; for example, an ACQ user may specify that an AC returned should be related to some specific keywords like ``research'' and ``sports''.

To enable efficient AC search, we develop the CL-tree index structure and three algorithms based on it. We evaluate our solutions on four large graphs, namely Flickr, DBLP, Tencent, and DBpedia. Our results show that ACs are more effective and efficient than existing community retrieval approaches. Moreover, an AC contains more precise and personalized information than that of existing community search and detection methods.


% Reynold old version (before Mar 1)
%Given a graph $G$ and a vertex $q \in G$, the {\it community search query} (CSQ) returns a subgraph of $G$ that contains vertices related to $q$.  Communities are prevalent in social networks, bibliographical graphs, and knowledge bases, and can be used in applications such as product advertisement and setting up of social events.
%In this paper, we investigate CSQ for an {\it attributed graph}, whose vertices are associated with keywords. The {\it attributed community} (AC) so found is a subgraph of $G$, whose vertices are tightly connected and share common keywords.  The AC enables a better understanding of how and why a community is formed (e.g., members of an AC have a common interest in music, because they all have the same keyword ``music'').  We also allow an AC to be ``personalized''; for example, a CSQ user may specify that an AC returned must contain vertices that have the keyword ``sports''.
%
%%be contained in the AC.
%
%To enable efficient AC search, we develop the CL-tree structure and three related algorithms. We evaluate our solutions on four large graphs, namely Flickr, DBLP, Tencent, and DBpedia. Our results show that ACs are more efficient than existing community detection approaches. Moreover, an AC contains more precise and personalized information than existing community search and detection methods.

% Yixiang
%Communities are prevalent in social networks, bibliographical graphs, and knowledge bases, and they enable emerging applications like product advertisement and setting up of social events. Recently, the topic of \emph{online community search} has captured a lot of attention.  Given a vertex $q$ of a graph, the goal of {\it community search} is to find a community, a subgraph which contains vertices closely related to $q$. In this paper, we investigate the online search of communities from attributed graphs, where vertices have meaningful attributes or content. Particularly, we propose the \emph{label-aware community} (or LAC), which not just requires vertices to be structurally close to each other, but also needs to have common labels. The LAC allows a better understanding of how and why a community is formed (e.g., members of an LAC have a common interest in yoga, because they all have the same label ``music''). To enable efficient LAC search, we develop an index called the CL-tree, and investigate novel query algorithms based on it. We evaluate our solutions on four large graphs, namely Flickr, DBLP, Tencent, and DBpedia. Our experimental results show that LACs can be searched efficiently. Moreover, they reflect more precise meanings than communities found by existing community search and detection approaches.

